\documentclass[a4paper,12pt]{article}

\RequirePackage{epsfig}

\setlength\hoffset{-0.5in}      %% these work quite well with a 12pt font
\setlength\voffset{-0.5in}
\setlength{\textwidth}{6.30in}
\setlength{\textheight}{9.0in}

\bibliographystyle{unsrt}

\begin{document}

\begin{center}
{\Large\bf{Planning the Seating Chart for Optimal Weddings}} \\
      \vspace{5.0mm}
{\Large\bf{Project Plan}} \\
      \vspace{8mm}
      {\large\bf{Daniel Joffe}}  \\
      \vspace{5.0mm}
       {\tt u03dj17@abdn.ac.uk} \\
      \vspace{5.0mm}
      {\em Department of Computing Science,\\
       University of Aberdeen, Aberdeen AB24 3UE, UK} 
\end{center}


\section*{Introduction}
\iffalse
This section should briefly describe the background to your
project and explain why your project is a worthwhile task.
You could add some references here, if appropriate, to cite 
relevant articles
\cite{wooldridge2002,shoham95} and \cite{garcia-camino2005}.
\fi

% Framing the situation
A staple part of weddings is the dinner, during which guests
are grouped together and sat at tables for the duration of
the meal. Most people would prefer to eat with close friends
and family rather than with strangers or bitter enemies. In
the latter case, it may not be only those sat together who have
a bad time, but everyone within earshot as well.

% The problem
In weddings with upwards of a hundred guests, finding an
arrangement that makes everyone happy can be a difficult task.
Care must be taken to ensure that everyone is sat with as many
friends as possible and away from any nemeses they might have.
Matters are complicated further when two people who hate each
other have a mutual friend. Many hours have been spent wrangling
complex relationship networks into an amenable seating chart and
the failure to do so has ruined many weddings.

% My goal
The purpose of this project is to automate the process, so that
good seating charts can be found faster than humanly possible and
so that wedding planners can be assured knowing that their seating
plan is (among) the best possible for their particular problem
instance.

% Other applications
Though framed in the language of weddings, the underlying problem
is relevant to other areas. In universities, professors may wish
to find a composition of teams for a group project that results in
minimal in-fighting. A safari park may want to find some way of
grouping different animals together such that minimal land needs
to be bought, while maintaining that none of them eat each other.

\section*{Goals}

\iffalse
This section should describe the the main goals of your project.
In other words, describe {\em what} it is that you want to do.
Are you building a tool or an application? What functionality 
already exists, and what will you have to do yourself?

Try to make it clear which goals are central to your project, and which
might be optional extras. Try to be realistic about making your
goals {\em achievable.}
\fi

The ultimate goal is to find an efficient heuristic for solving the seating
chart problem and to assess its feasibility in terms of correctness and
computational resources. To this end, I intend to verify the results of Meghan
L. Bellows and J. D. Luc Peterson (cite this) and compare their approach to
standard metaheuristics such as genetic algorithms and hill climbing.

As an additional goal, it would be desirable to test these same approaches on a
more refined model of the problem, such as accounting for  variably-sized dinner
tables or more complex relationships where three people can be friends pairwise,
but cause trouble when all put together. Another possibility is that the happy
couple invite a guest they don't like very much, and want everyone except them to
have a pleasant meal.


\section*{Methodology}

This section should describe {\em how} you will conduct
your project. You should explain in general terms
the activities you will be carrying out during your project, such as:
%
\begin{itemize}
\item reading about related work -- either to get ideas on how to
      proceed, or to compare your approach with what was done before;
\item learing a new programming language or API;
\item learning about relevant technologies;
\item developing prototypes to test ideas;
\item testing and debugging early design choices.
\end{itemize}
%
The above examples are purely suggestions. You should try to think
of what would be appropriate for your specific project.

\section*{Resources Required}

You should mention here the hardware and software resources you will
require. Even if it seems obvious that you might only need Java and a PC, 
you should still say so!


\section*{Risk Assessment}

Try to describe possible circumstances (e.g. a particular piece of
technology doesn't work or is too expensive) that might cause
the project to become become infeasible. What would you have to do
or change to recover your project?

\section*{Timetable}

This section should describe the {\em schedule} for your project. 
You should describe the various activities you expect to perform
and their durations, along with any deadlines and deliverables.
It is often useful to collect all of this information in a
{\em Gantt chart}, as shown in Figure~\ref{fig:plan} below:

\begin{figure}[htb]
\begin{center}
\includegraphics[scale=0.65]{ProjectPlan.eps}
\caption{Main Project Activities\label{fig:plan}}
\end{center}
\end{figure}

% For information, this figure was created using xfig, and 
% converted to encapsulated postscript by a command in the Makefile
% before being included as a graphic in the \LaTeX document.

Don't forget to add time at the end of your project for 
evaluation and writing-up! This could easily require 2-3 weeks.


\bibliography{ProjectPlan}

\end{document}
